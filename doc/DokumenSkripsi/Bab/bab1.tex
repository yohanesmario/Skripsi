\chapter{Pendahuluan}
\label{chap:pendahuluan}



\section{Latar Belakang}
\label{sec:latar_belakang}

Internet adalah salah satu hal yang sulit dipisahkan dari keseharian manusia masa kini. Salah satu cara seseorang dapat mengakses internet adalah dengan menggunakan teknologi Wi-Fi. Wi-Fi mengharuskan pengguna terhubung pada suatu \textit{access-point}. \textit{Access-point} tersebut dapat memiliki dua status, yaitu terproteksi atau tidak terproteksi. Proteksi pada \textit{access-point} dapat dilakukan dengan beberapa cara, yaitu menggunakan protokol IEEE 802.11, atau menggunakan \textit{captive portal}. Alat yang sudah pernah terhubung dengan \textit{access-point} yang diproteksi dengan protokol IEEE 802.11 akan dengan mudah terhubung kembali dengan \textit{access-point} tersebut karena alat tersebut biasanya sudah menyimpan password untuk \textit{access-point} yang bersangkutan. Alat yang akan terhubung dengan \textit{access-point} yang diproteksi menggunakan \textit{captive portal} belum memiliki cara untuk mengingat username dan password untuk \textit{captive portal} tersebut sehingga login otomatis belum dapat dilakukan untuk \textit{access-point} jenis ini.

Berdasarkan pengamatan peneliti, \textit{captive portal} banyak digunakan untuk proteksi \textit{access point} pada tempat-tempat umum seperti lingkungan universitas, \textit{starbucks}, \textit{Mc Donald's}, dan beberapa tempat yang dapat diakses melalui \textit{@wifi.id}, \textit{free@wifi.id} dan \textit{access point} sejenis. Oleh karena itu, dibutuhkan mekanisme yang bisa membantu proses login untuk \textit{access-point} tipe ini. Terdapat dua cara untuk menciptakan mekanisme ini, yaitu dengan mengintegrasikannya dengan sistem operasi, atau menggunakan perangkat lunak pihak ketiga. Untuk dapat melakukan pengintegrasian mekanisme tersebut dengan sistem operasi, dibutuhkan akses kepada kode sumber sistem aplikasi tersebut. Oleh karena itu, pilihan yang lebih bijak sebagai seseorang yang tidak memiliki akses tersebut adalah dengan menciptakan perangkat lunak pihak ketiga.



\section{Rumusan Masalah}
\label{sec:rumusan_masalah}

Rumusan masalah yang akan dibahas pada penelitian ini adalah sebagai berikut:

\begin{itemize}
	\item{Bagaimana caranya melakukan implementasi login otomatis pada \textit{captive portal} yang memiliki tingkat kenyamanan yang setara dengan login otomatis pada proteksi Wi-Fi berbasis protokol IEEE 802.11?}
	\item{Apa saja yang perlu dilakukan untuk mengamankan username dan password yang disimpan oleh user?}
	\item{Informasi apa saja yang dibutuhkan untuk menciptakan identitas unik untuk setiap \textit{captive portal} pada jaringan yang berbeda?}
\end{itemize}



\section{Tujuan Penelitian}
\label{sec:tujuan_penelitian}

Tujuan penelitian ini adalah sebagai berikut:

\begin{itemize}
	\item{Melakukan implementasi login otomatis pada \textit{captive portal} yang memiliki tingkat kenyamanan yang setara dengan login otomatis pada proteksi Wi-Fi berbasis protokol IEEE 802.11.}
	\item{Memastikan username dan password pengguna disimpan secara aman?}
	\item{Menentukan informasi yang dibutuhkan untuk menciptakan identitas unik untuk setiap \textit{captive portal} pada jaringan yang berbeda?}
\end{itemize}



\section{Batasan Masalah}
\label{sec:batasan_masalah}

Batasan masalah dari penelitian ini adalah sebagai berikut:

\begin{itemize}
	\item{Perangkat lunak dibangun untuk sistem operasi Windows 8 sampai dengan Windows 10.}
    \item{Perangkat lunak dibangun menggunakan bahasa pemrograman C\#.}
	\item{Elemen keamanan informasi yang akan diimplementasikan pada perangkat lunak ini adalah enkripsi username dan password yang disimpan oleh user.}
\end{itemize}



\section{Metodologi Penelitian}
\label{sec:metodologi_penelitian}

Metodologi penelitian yang akan dilakukan adalah sebagai berikut:

\begin{enumerate}
    \item{Melakukan studi literatur mengenai hal-hal yang berkaitan dengan perancangan dan pembuatan aplikasi, yaitu:}
        \subitem{Pemrograman menggunakan bahasa pemrograman C\#.}
        \subitem{Penggunaan kelas WebBrowser pada C\#.}
        \subitem{Penggunaan objek PasswordVault pada C\#.}
        \subitem{Sistem login Wi-Fi berbasis web pada \textit{access-point} dari berbagai merek.}
    \item{Melakukan analisis perangkat lunak sejenis.}
    \item{Melakukan analisis kebutuhan untuk mengimplementasikan mekanisme login otomatis ini.}
    \item{Merancang perangkat lunak login otomatis ini.}
    \item{Melakukan implementasi hasil rancangan dengan bahasa pemrograman C\# pada sistem operasi Windows 10.}
    \item{Melakukan pengujian terhadap perangkat lunak untuk menghasilkan perbaikan teradap perangkat lunak tersebut.}
    \item{Membuat kesimpulan dari hasil penelitian dan saran untuk penelitian selanjutnya.}
\end{enumerate}


\section{Sistematika Pembahasan}
\label{sec:sistematika_pembahasan}

Laporan skripsi ini terdiri dari beberapa bab, yaitu:

\begin{enumerate}
    \item{Bab Pendahuluan \\ Bab ini berisi latar belakang, rumusan masalah, tujuan penelitian, batasan masalah, metodologi penelitian, dan sistematika pembahasan.}
    \item{Bab Dasar Teori \\ Bab ini berisi dasar-dasar pemrograman menggunakan bahasa pemrograman C\#, dokumentasi kelas WebBrowser, dokumentasi objek PasswordVault, sistem login pada Mikrotik, dan sistem login pada Cisco.}
    \item{Bab Analisis \\ Bab ini berisi analisis kebutuhan untuk perancangan dan pembuatan perangkat lunak login otomatis untuk proteksi Wi-Fi berbasis web.}
    \item{Bab Perancangan \\ Bab ini berisi perancangan perangkat lunak login otomatis untuk proteksi Wi-Fi berbasis web.}
    \item{Bab Implementasi dan Pengujian \\ Bab ini berisi implementasi perangkat lunak login otomatis untuk proteksi Wi-Fi berbasis web beserta pengujian dan hasil perbaikannya.}
    \item{Bab Kesimpulan dan Saran \\ Bab ini berisi kesimpulan dari penelitian ini dan saran yang akan diberikan untuk penelitian selanjutnya.}
\end{enumerate}
