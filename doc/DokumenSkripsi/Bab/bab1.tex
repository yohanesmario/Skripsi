\chapter{Pendahuluan}
\label{chap:pendahuluan}

Bab ini menjelaskan mengenai latar belakang, rumusan masalah, tujuan penelitian, batasan masalah, metodologi penelitian, dan sistematika pembahasan.



\section{Latar Belakang}
\label{sec:latar_belakang}

Internet adalah salah satu hal yang sulit dipisahkan dari keseharian manusia masa kini. Salah satu cara seseorang dapat mengakses internet adalah dengan menggunakan teknologi WiFi. WiFi mengharuskan pengguna terhubung pada suatu \textit{access point}. \textit{Access point} tersebut dapat memiliki dua status, yaitu terproteksi atau tidak terproteksi. Proteksi pada \textit{access point} dapat dilakukan dengan beberapa cara, yaitu menggunakan protokol IEEE 802.11\cite{IEEE80211:2011}, atau menggunakan \textit{captive portal}. Alat yang sudah pernah terhubung dengan \textit{access point} yang diproteksi dengan protokol IEEE 802.11 akan dengan mudah terhubung kembali dengan \textit{access point} tersebut karena alat tersebut biasanya sudah menyimpan \textit{password} untuk \textit{access point} yang bersangkutan. Alat yang akan terhubung dengan \textit{access point} yang diproteksi menggunakan \textit{captive portal} belum memiliki cara untuk mengingat \textit{username} dan \textit{password} untuk \textit{captive portal} tersebut sehingga login otomatis belum dapat dilakukan untuk \textit{access point} jenis ini.

\textit{Captive portal} banyak digunakan untuk proteksi \textit{access point} pada tempat-tempat umum seperti lingkungan universitas, \textit{starbucks}, \textit{McDonald's}, dan beberapa tempat yang dapat diakses melalui \textit{@wifi.id}, \textit{free@wifi.id} dan \textit{access point} sejenis. Oleh karena itu, dibutuhkan mekanisme yang bisa membantu proses login untuk \textit{access point} tipe ini. Terdapat dua cara untuk menciptakan mekanisme ini, yaitu dengan mengintegrasikannya dengan sistem operasi, atau menggunakan perangkat lunak pihak ketiga. Untuk dapat melakukan pengintegrasian mekanisme tersebut dengan sistem operasi, dibutuhkan akses kepada kode sumber sistem operasi tersebut. Oleh karena itu, pilihan yang lebih bijak sebagai seseorang yang tidak memiliki akses tersebut adalah dengan menciptakan perangkat lunak pihak ketiga.

Perangkat lunak yang dibuat harus mampu menyimpan informasi login dan mengirimkannya kembali secara otomatis kepada \textit{captive portal}. Perangkat lunak tersebut juga harus mampu mengamankan informasi login yang sudah disimpan agar tidak dapat diakses oleh sembarang orang. Perangkat lunak ini juga harus mampu melakukan identifikasi \textit{captive portal} yang berbeda agar tidak salah memasukkan informasi login.



\section{Rumusan Masalah}
\label{sec:rumusan_masalah}

Rumusan masalah yang dibahas pada penelitian ini adalah sebagai berikut:

\begin{enumerate}
	\item{Bagaimana caranya melakukan implementasi login otomatis pada \textit{captive portal} yang memiliki tingkat kenyamanan yang setara dengan login otomatis pada proteksi WiFi berbasis protokol IEEE 802.11?}
	\item{Apa saja yang perlu dilakukan untuk mengamankan \textit{username} dan \textit{password} yang disimpan oleh user?}
	\item{Informasi apa saja yang dibutuhkan untuk menciptakan identitas unik untuk setiap \textit{captive portal} pada jaringan yang berbeda?}
\end{enumerate}



\section{Tujuan Penelitian}
\label{sec:tujuan_penelitian}

Tujuan penelitian ini adalah sebagai berikut:

\begin{enumerate}
	\item{Melakukan implementasi login otomatis pada \textit{captive portal} yang memiliki tingkat kenyamanan yang setara dengan login otomatis pada proteksi WiFi berbasis protokol IEEE 802.11.}
	\item{Memastikan \textit{username} dan \textit{password} pengguna disimpan secara aman.}
	\item{Menentukan informasi yang dibutuhkan untuk menciptakan identitas unik untuk setiap \textit{captive portal} pada jaringan yang berbeda.}
\end{enumerate}



\section{Batasan Masalah}
\label{sec:batasan_masalah}

Batasan masalah dari penelitian ini adalah sebagai berikut:

\begin{enumerate}
	\item{Perangkat lunak dibangun untuk sistem operasi Windows 8 sampai dengan Windows 10.}
    \item{Perangkat lunak dibangun menggunakan bahasa pemrograman C\#.}
	\item{Elemen keamanan informasi yang diimplementasikan pada perangkat lunak ini adalah enkripsi \textit{username} dan \textit{password} yang disimpan oleh user.}
\end{enumerate}



\section{Metodologi Penelitian}
\label{sec:metodologi_penelitian}

Metodologi penelitian yang dilakukan pada penelitian ini adalah sebagai berikut:

\begin{enumerate}
    \item{Melakukan studi literatur mengenai hal-hal yang berkaitan dengan perancangan dan pembuatan aplikasi, yaitu:}
        \subitem{Cara kerja dan protokol-protokol yang terkait dengan \textit{captive portal}.}
        \subitem{Pemrograman menggunakan \textit{.NET framework}.}
        \subitem{\textit{Universal Windows Platform} (UWP).}
        \subitem{Penggunaan kelas WebBrowser pada C\#.}
        \subitem{Penggunaan objek PasswordVault pada C\#.}
    \item{Melakukan analisis perangkat lunak sejenis.}
    \item{Melakukan analisis kebutuhan untuk mengimplementasikan mekanisme login otomatis ini.}
    \item{Merancang perangkat lunak login otomatis ini.}
    \item{Melakukan implementasi hasil rancangan dengan bahasa pemrograman C\# pada sistem operasi Windows 10.}
    \item{Melakukan pengujian terhadap perangkat lunak untuk menghasilkan perbaikan teradap perangkat lunak tersebut.}
    \item{Membuat kesimpulan dari hasil penelitian dan saran untuk penelitian selanjutnya.}
\end{enumerate}


\section{Sistematika Pembahasan}
\label{sec:sistematika_pembahasan}

Laporan skripsi ini terdiri dari beberapa bab, yaitu:

\begin{enumerate}
    \item{Bab Pendahuluan \\ Bab ini berisi latar belakang, rumusan masalah, tujuan penelitian, batasan masalah, metodologi penelitian, dan sistematika pembahasan.}
    \item{Bab Dasar Teori \\ Bab ini berisi dasar-dasar teori dasar mengenai \textit{captive portal}, \textit{.NET framework}, \textit{Universal Windows Platform} (UWP), dokumentasi kelas WebBrowser dan dokumentasi objek PasswordVault.}
    \item{Bab Analisis \\ Bab ini berisi analisis kebutuhan untuk perancangan dan pembuatan perangkat lunak login otomatis untuk proteksi WiFi berbasis web.}
    \item{Bab Perancangan \\ Bab ini berisi perancangan perangkat lunak login otomatis untuk proteksi WiFi berbasis web.}
    \item{Bab Implementasi dan Pengujian \\ Bab ini berisi implementasi perangkat lunak login otomatis untuk proteksi WiFi berbasis web beserta pengujian dan hasil perbaikannya.}
    \item{Bab Kesimpulan dan Saran \\ Bab ini berisi kesimpulan dari penelitian ini dan saran yang diberikan untuk penelitian selanjutnya.}
\end{enumerate}
