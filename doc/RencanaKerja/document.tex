\documentclass[a4paper,twoside]{article}
\usepackage[T1]{fontenc}
\usepackage[bahasa]{babel}
\usepackage{graphicx}
\usepackage{graphics}
\usepackage{float}
\usepackage[cm]{fullpage}
\pagestyle{myheadings}
\usepackage{etoolbox}
\usepackage{setspace} 
\usepackage{lipsum} 
\setlength{\headsep}{30pt}
\usepackage[inner=2cm,outer=2.5cm,top=2.5cm,bottom=2cm]{geometry} %margin
% \pagestyle{empty}

\makeatletter
\renewcommand{\@maketitle} {\begin{center} {\LARGE \textbf{ \textsc{\@title}} \par} \bigskip {\large \textbf{\textsc{\@author}} }\end{center} }
\renewcommand{\thispagestyle}[1]{}
\markright{\textbf{\textsc{AIF401/AIF402 \textemdash Rencana Kerja Skripsi \textemdash Sem. Genap 2015/2016}}}

\onehalfspacing
 
\begin{document}

\title{\@judultopik}
\author{\nama \textendash \@npm} 

%tulis nama dan NPM anda di sini:
\newcommand{\nama}{Yohanes Mario Chandra}
\newcommand{\@npm}{2011730031}
\newcommand{\@judultopik}{Wi-Fi Web Auto Login} % Judul/topik anda
\newcommand{\jumpemb}{1} % Jumlah pembimbing, 1 atau 2
\newcommand{\tanggal}{08/09/2016}

% Dokumen hasil template ini harus dicetak bolak-balik !!!!

\maketitle

\pagenumbering{arabic}

\section{Deskripsi}
%Tuliskan deskripsi dari topik skripsi yang akan anda ajukan. Di sini dapat dituliskan latar belakang, seperti apa penelitian yang sudah ada sebelumnya dan apa yang akan anda kerjakan. Sertakan gambar agar penjelasan anda menjadi lebih baik.
%
%Pada skripsi ini, akan dibuat sebuah perangkat lunak yang dapat menampilkan visualisasi dan simulasi kerumunan orang yang berkunjung ke sebuah museum. Dengan menggunakan perangkat lunak tersebut, pengelola museum dapat mengatur tempat peletakan objek sehingga tidak terjadi kerumunan yang terlalu padat.
%
%Dari berbagai macam teknik yang dapat digunakan untuk melakukan simulasi kerumunan, dipilih dua buah teknik yaitu teknik {\it flow tiles} dan {\it social force model (steering behaviour)}.
%
%Dst, dst, dst, \ldots\ldots\ldots 
%
%Perangkat lunak akan dibuat dengan bantuan {\it framework} OpenSteer. Sebagai studi kasus, museum yang digunakan untuk melakukan simulasi adalah Museum Geologi Bandung.
%
%Dst, dst, dst, \ldots\ldots\ldots 

Internet adalah salah satu hal yang sulit dipisahkan dari keseharian manusia masa kini. Salah satu cara seseorang dapat mengakses internet adalah dengan menggunakan teknologi Wi-Fi. Wi-Fi mengharuskan pengguna terhubung pada suatu \textit{access point}. \textit{Access point} tersebut dapat memiliki dua status, yaitu terproteksi atau tidak terproteksi. Proteksi pada \textit{access point} dapat dilakukan dengan beberapa cara, yaitu menggunakan protokol IEEE 802.11, atau menggunakan \textit{captive portal}. Alat yang sudah pernah terhubung dengan \textit{access point} yang diproteksi dengan protokol IEEE 802.11 akan dengan mudah terhubung kembali dengan \textit{access point} tersebut karena alat tersebut biasanya sudah menyimpan \textit{password} untuk \textit{access point} yang bersangkutan. Alat yang akan terhubung dengan \textit{access point} yang diproteksi menggunakan \textit{captive portal} belum memiliki cara untuk mengingat \textit{username} dan \textit{password} untuk \textit{captive portal} tersebut sehingga login otomatis belum dapat dilakukan untuk \textit{access point} jenis ini.

Berdasarkan pengamatan peneliti, \textit{captive portal} banyak digunakan untuk proteksi \textit{access point} pada tempat-tempat umum seperti lingkungan universitas, \textit{starbucks}, \textit{McDonald's}, dan beberapa tempat yang dapat diakses melalui \textit{@wifi.id}, \textit{free@wifi.id} dan \textit{access point} sejenis. Oleh karena itu, dibutuhkan mekanisme yang bisa membantu proses login untuk \textit{access point} tipe ini. Terdapat dua cara untuk menciptakan mekanisme ini, yaitu dengan mengintegrasikannya dengan sistem operasi, atau menggunakan perangkat lunak pihak ketiga. Untuk dapat melakukan pengintegrasian mekanisme tersebut dengan sistem operasi, dibutuhkan akses kepada kode sumber sistem aplikasi tersebut. Oleh karena itu, pilihan yang lebih bijak sebagai seseorang yang tidak memiliki akses tersebut adalah dengan menciptakan perangkat lunak pihak ketiga.

Perangkat lunak dibuat pada platform Windows dengan bahasa dan \textit{library} C\#. Alasan penggunaan C\# adalah untuk mendapatkan akses ke sistem internal windows secara menyeluruh.

\section{Rumusan Masalah}
Rumusan masalah yang dibahas pada penelitian ini adalah sebagai berikut:

\begin{itemize}
	\item{Bagaimana caranya melakukan implementasi login otomatis pada \textit{captive portal} yang memiliki tingkat kenyamanan yang setara dengan login otomatis pada proteksi Wi-Fi berbasis protokol IEEE 802.11?}
	\item{Apa saja yang perlu dilakukan untuk mengamankan \textit{username} dan \textit{password} yang disimpan oleh user?}
	\item{Informasi apa saja yang dibutuhkan untuk menciptakan identitas unik untuk setiap \textit{captive portal} pada jaringan yang berbeda?}
\end{itemize}

\section{Tujuan}
\label{sec:tujuan_penelitian}

Tujuan penelitian ini adalah sebagai berikut:

\begin{itemize}
	\item{Melakukan implementasi login otomatis pada \textit{captive portal} yang memiliki tingkat kenyamanan yang setara dengan login otomatis pada proteksi Wi-Fi berbasis protokol IEEE 802.11.}
	\item{Memastikan \textit{username} dan \textit{password} pengguna disimpan secara aman?}
	\item{Menentukan informasi yang dibutuhkan untuk menciptakan identitas unik untuk setiap \textit{captive portal} pada jaringan yang berbeda?}
\end{itemize}

\section{Deskripsi Perangkat Lunak}

Perangkat lunak akhir yang akan dibuat memiliki fitur minimal sebagai berikut:
\begin{itemize}
	\item Perangkat lunak mampu mendeteksi jaringan Wi-Fi yang tidak langsung terhubung dengan internet.
	\item Pengguna dapat melakukan login pada \textit{captive portal} seperti biasa dan perangkat lunak dapat merekam kredensial (username dan password) dan aksi-aksi pengguna.
	\item Perangkat lunak dapat melakukan login otomatis pada jaringan Wi-Fi yang sudah memiliki kredensial.
	\item Perangkat lunak dapat menyimpan kredensial pengguna dengan aman.
	\item Pengguna dapat memilih untuk tidak menggunakan kredensial yang sudah disimpan pada saat akan melakukan login otomatis pada captive portal.
		
\end{itemize}

\section{Detail Pengerjaan Skripsi}

Bagian-bagian pekerjaan skripsi ini adalah sebagai berikut :
	\begin{enumerate}
        \item{Melakukan studi literatur mengenai hal-hal yang berkaitan dengan perancangan dan pembuatan aplikasi, yaitu:}
            \subitem{Cara kerja dan protokol-protokol yang terkait dengan \textit{captive portal}.}
            \subitem{Pemrograman menggunakan \textit{.NET framework}.}
            \subitem{\textit{Universal Windows Platform} (UWP).}
            \subitem{Penggunaan kelas WebBrowser pada C\#.}
            \subitem{Penggunaan objek PasswordVault pada C\#.}
        \item{Melakukan analisis perangkat lunak sejenis.}
        \item{Melakukan analisis kebutuhan untuk mengimplementasikan mekanisme login otomatis ini.}
        \item{Merancang perangkat lunak login otomatis ini.}
        \item{Melakukan implementasi hasil rancangan dengan bahasa pemrograman C\# pada sistem operasi Windows 10.}
        \item{Melakukan pengujian terhadap perangkat lunak untuk menghasilkan perbaikan teradap perangkat lunak tersebut.}
        \item{Membuat kesimpulan dari hasil penelitian dan saran untuk penelitian selanjutnya.}
        \item{Menulis dokumen skripsi.}
    \end{enumerate}

\section{Rencana Kerja}

\begin{center}
  \begin{tabular}{ | c | c | c | c | l |}
    \hline
    1*  & 2*(\%) & 3*(\%) & 4*(\%) &5*\\ \hline \hline
    1   & 5  & 5  &  &  \\ \hline
    2   & 10 & 10  &   & \\ \hline
    3   & 15  & 10  & 5 & {\footnotesize kaji ulang singkat kebutuhan pada S2}  \\ \hline
    4   & 15  & 10  & 5 & {\footnotesize perubahan rancangan pada S2 jika ada kebutuhan tambahan} \\ \hline
    5   & 25  &   & 25 & \\ \hline
    6   & 10 &   & 10  & \\ \hline
    7   & 20  & 5  & 15 &  {\footnotesize sebagian bab 1 dan 2, serta bagian awal analisis di S1}\\ \hline
    Total  & 100  & 40  & 60 &  \\ \hline
                          \end{tabular}
\end{center}

Keterangan (*)\\
1 : Bagian pengerjaan Skripsi (nomor disesuaikan dengan detail pengerjaan di bagian 5)\\
2 : Persentase total \\
3 : Persentase yang akan diselesaikan di Skripsi 1 \\
4 : Persentase yang akan diselesaikan di Skripsi 2 \\
5 : Penjelasan singkat apa yang dilakukan di S1 (Skripsi 1) atau S2 (skripsi 2)

\vspace{1cm}
\centering Bandung, \tanggal\\
\vspace{2cm} \nama \\ 
\vspace{1cm}

Menyetujui, \\
\ifdefstring{\jumpemb}{2}{
\vspace{1.5cm}
\begin{centering} Menyetujui,\\ \end{centering} \vspace{0.75cm}
\begin{minipage}[b]{0.45\linewidth}
% \centering Bandung, \makebox[0.5cm]{\hrulefill}/\makebox[0.5cm]{\hrulefill}/2013 \\
\vspace{2cm} Nama: \makebox[3cm]{\hrulefill}\\ Pembimbing Utama
\end{minipage} \hspace{0.5cm}
\begin{minipage}[b]{0.45\linewidth}
% \centering Bandung, \makebox[0.5cm]{\hrulefill}/\makebox[0.5cm]{\hrulefill}/2013\\
\vspace{2cm} Nama: \makebox[3cm]{\hrulefill}\\ Pembimbing Pendamping
\end{minipage}
\vspace{0.5cm}
}{
% \centering Bandung, \makebox[0.5cm]{\hrulefill}/\makebox[0.5cm]{\hrulefill}/2013\\
\vspace{2cm} Nama: \makebox[3cm]{\hrulefill}\\ Pembimbing Tunggal
}
\end{document}

